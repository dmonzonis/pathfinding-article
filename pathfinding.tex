\documentclass[12pt]{report}
\usepackage{amsthm}
\usepackage{amssymb}
\usepackage{amsmath}
\usepackage[parfill]{parskip}  				% Newlines create actual paragraphs
\usepackage{algorithm} 						% For pseudocode
\usepackage{algpseudocode}  					% For pseudocode
\usepackage{hyperref}						% Adds hyperlinks
\usepackage{bm}								% Bold math
\usepackage{tikz}							% Graph drawing
\usepackage{subcaption}						% Subfigures
\usepackage[toc]{appendix}					% Appendices

% Add spacing between theorems, which is zeroed by parskip
\makeatletter
\def\thm@space@setup{%
  \thm@preskip=\parskip \thm@postskip=0pt
}
\makeatother

% Better looking empty set
\let\oldemptyset\emptyset
\let\emptyset\varnothing

% algpseudocode modifications
\renewcommand{\algorithmicrequire}{\textbf{Input:}}
\renewcommand{\algorithmicensure}{\textbf{Output:}}
\algnewcommand{\Continue}{\State \textbf{continue}}

% Theorem keyword definitions
\newtheorem{theorem}{Theorem}[chapter]
\newtheorem{definition}[theorem]{Definition}
\newtheorem{proposition}[theorem]{Proposition}
\newtheorem{lemma}[theorem]{Lemma}
\newtheorem{corollary}[theorem]{Corollary}

% DEBUG commands
\newcommand{\DEBUGComment}[1]{{\color{red}{#1}}}
\newcommand{\DEBUGQuestion}[1]{{\color{blue}{#1}}}

\title{
{Pathfinding algorithms in graphs and applications}\\
{\large Universitat de Barcelona}
}
\author{Daniel Monzonís Laparra}

\begin{document}
\maketitle

\chapter*{Abstract}
Abstract goes here.

\tableofcontents

\chapter{Introduction}
Many problems in the fields of science, mathematics and engineering can be generalised to the problem of finding a path in a graph. Examples of such problems include routing of telephone or Internet traffic, layout of printed circuit boards, automated theorem proving, GPS routing, decision making in artificial intelligence and robotics. In this article, we will provide a mathematical and applied approach to this problem.

For most of the examples in the article we will be using a grid graph, represented by a tiled 2D map, in which each tile (or node) is connected to its adjacent tiles. In the map, each tile can have a different weight, or be a wall, which is impassable. This notion of weighted nodes and walls can be translated to the structure of a weighted directed graph by thinking of the weights of the tiles as the weights of the edges that connect adjacent tiles to it. Walls can be thought of as nodes that are not connected (or just non-existent) to any other node, and therefore not reachable from other nodes. A pathfinding simulator was built for this article to help illustrate these examples.

\chapter{The $A^*$ algorithm}
The $A^*$ algorithm is a very popular algorithm used in many applications to find optimal paths between points. The algorithm works on graphs, a structure well studied in graph theory.

\begin{definition}
A \textbf{directed graph} $G$ is a pair of sets $(V, E)$, where $V$ is the set of vertices or nodes, and $E$ the set of edges, formed by pairs of vertices.
\end{definition}

\begin{definition}
A \textbf{weighted directed graph} is a graph in which $\forall e \in E \ \exists w(e) \in \mathbb{R}$. We call $w(e)$ the \textbf{weight} or \textbf{cost} of the edge $e$.

If $e = (u, v)$ for some $u, v \in V$, then we equivalently call $d(u, v) = w(e)$ the \textbf{distance} from $u$ to $v$.
\end{definition}

From now on, when we talk about a graph, we will implicitly refer to a finite weighted directed graph, since it's the most general type of graph we will work with. An unweighted graph can be thought of as a weighted graph where all weights are equal to one, and an undirected graph as a directed graph where $\forall (u, v) \in E,\  \exists (v, u) \in E$. Also, the graphs we will work with are finite, meaning that $|V| < \infty$ and $|E| < \infty$.

\begin{definition}[Path]
Given a graph $G$, and $u, v \in V$, a \textbf{path} $P$ between $u$ and $v$ is an ordered list of a certain amount of edges, $N$, in the form

\[ P = \{(u,v_1), (v_1, v_2), \dots, (v_{N-2}, v_{N-1}), (v_{N-1}, v)\} \]
\end{definition}

Note that paths are not unique. There may exist multiple paths between two nodes.

\begin{definition}
We say that two nodes $u, v \in V$ are connected $\Longleftrightarrow$ There exists a path $P$ between $u$ and $v$.
\end{definition}

\begin{definition}
Given a path $P$ between two nodes $u, v \in V$, given a node $n$ which is in the path, we say that $n'$ is the \textbf{successor} of $n$ if $(n, n') \in P$, that is, if we followed the path $P$, the next node we would visit when we have reached $n$ would be $n'$.
\end{definition}

\begin{definition}
If $u \in V$, we define the \textbf{connected component} of $u$ as
\[ C_u = \{ v \in V\ |\ u \ \text{and} \ v \ \text{are connected}\  \} \]
which is a subgraph of $G$.
\end{definition}

\begin{definition}[Distance of a path]
Given a path $P$ in a weighted directed graph $G$, we define the \textbf{weight} or \textbf{distance} of the path, $dist(P)$, as

\[ d(P) = \sum_{e \in P} w(e) \]

If $P$ is a path between $u$ and $v$, we can equivalently write

\[ d_P(u, v) = d(P) \]

If the context is clear, we will just write $d(u, v)$. For convenience, if $u$ and $v$ are not connected, we define the distance between them as $d(u, v) = \infty$.
\end{definition}

Note that, in general, $d(u, v) \neq d(v, u)$, since graphs can be directed and not all paths may be reversible.

\begin{definition}[Optimal path]
\label{def:optimal}
Given a set of all the existing paths between two nodes $u$ and $v$, $\chi_{(u, v)}$, we will say that a path $P \in \chi_{(u,v)}$ is \textbf{optimal} if and only if $d(P) \le d(P') \ \forall P' \in \chi_{(u, v)}$.
\end{definition}

We will also say that $P$ is the \textbf{shortest distance path} if it is optimal, and we will write the distance that fulfils the condition in definition \ref{def:optimal} as $\delta(u,v)$. Clearly, in a finite graph, an optimal path between any two nodes always exists, since in this case $\chi_{(u, v)}$ is finite.

Instead of presenting the $A^*$ algorithm without any background, we will first briefly discuss some other widely known algorithms that can be used to find paths between nodes in graphs, and build up to the main algorithm by trying to gradually improve the performance.

\section{BFS}
Breadth-First Search, or BFS for short, attempts to find a path by methodically examining all the neighbours of each node it examines.

The algorithm uses a queue\footnote{See \ref{annex:queue} for more information on queues.} to keep track of the next nodes to examine, adding all unvisited neighbours of a node when it is examined, until the queue is empty. Explored nodes are kept in a set, so that we don't explore the same node twice. The set of nodes that haven't been explored yet is often called the \emph{open set}, and the set of visited nodes the \emph{closed set}. In the same way, nodes in the open set are called \emph{open nodes}, and nodes in the closed set are called \emph{closed nodes}.

The pseudocode for the algorithm is presented in algorithm \ref{alg:bfs}. The algorithm returns a map called \emph{previous}, which maps every node to the node we came from in the path that the algorithm computes. The procedure that we will use to reconstruct the path from this map for all algorithms is presented in algorithm \ref{alg:reconstruct}.

\begin{algorithm}
\caption{Breadth-First Search}
\label{alg:bfs}
\begin{algorithmic}[1]
\Procedure{BFS}{$G, \alpha, \beta$}
\Require Graph $G = (V, E)$, directed or undirected; source node $\alpha \in V$; goal node $\beta \in V$
\Ensure Given $u \in V$, previous[$u$] gives us the node come from to reach $u$ in the path computed
\State Q $\gets Queue()$
\State $S \gets Set()$ \Comment{Keeps track of explored nodes}
\State previous $\gets Map()$
\For {$u \in V$}
	\State previous[$u$] $\gets \varnothing$
\EndFor
\State Q.enqueue($\alpha$)
\State $S$.add($\alpha$)
\While {not Q.empty()}
	\State $u \gets$ Q.front()
	\State Q.dequeue()
	\For {v $\in$ $u$.neighbours()}
		\If {$v \not\in S$}
			\State Q.enqueue($v$)
			\State S.add($v$)
			\State previous[$v$] $\gets u$
		\EndIf
	\EndFor
\EndWhile
\State \Return ReconstructPath(previous, $\alpha$, $\beta$)
\EndProcedure
\end{algorithmic}
\end{algorithm}

\begin{algorithm}
\caption{Reconstruct path}
\label{alg:reconstruct}
\begin{algorithmic}[1]
\Procedure{ReconstructPath}{previous, $\alpha$, $\beta$}
\Require The map $previous$ returned by the pathfinding algorithm; source node $\alpha \in V$; goal node $\beta \in V$
\Ensure An ordered list $P$ with the nodes from the path from $\alpha$ to $\beta$
\State $P \gets$ []  \Comment{Empty array for the path}
\State $u \gets \beta$
\While {$u \neq \alpha$}
	\State $P$.push($u$)
	\State $u \gets$ previous[$\beta$]
\EndWhile
\State $P$.push($\alpha$)
\State \Return $P$.reversed()
\EndProcedure
\end{algorithmic}
\end{algorithm}

In the rest of this article, we will consider a set of goal nodes instead of a single goal node, since this makes the algorithms, and therefore the results shown, more general. We will call this set of goal nodes $T$. This way, having a single goal node is only a special case of the more general condition when $|T| = 1$.

We see that with this version of BFS, all nodes in the connected component of the starting node are explored. We can improve the performance if we halt the execution once we reach a goal node, which is a fair condition since all we're looking for is a path between the source and a goal node, and once we find it we don't need to keep searching. This new version of the algorithm with the early exit is shown in algorithm \ref{alg:bfs_early_exit}.

\begin{algorithm}
\caption{Breadth-First Search with early exit}
\label{alg:bfs_early_exit}
\begin{algorithmic}[1]
\Procedure{BFS}{$G, \alpha, T$}
\Require Graph $G = (V, E)$, directed or undirected; source node $\alpha \in V$; set of goal nodes $T \subset V$
\Ensure Given $u \in V$, previous[$u$] gives us the node come from to reach $u$ in the path computed, if it has been explored
\State Q $\gets Queue()$
\State $S \gets Set()$
\State previous $\gets Map()$
\For {$u \in V$}
	\State previous[$u$] $\gets \varnothing$
\EndFor
\State Q.enqueue($\alpha$)
\State $S$.add($\alpha$)
\While {not Q.empty()}
	\State $u \gets$ Q.front()
	\State Q.dequeue()
	\For {$v \in$ u.neighbours()}
		\If {$v \not\in$ S}
			\If {$v \in T$} \Comment{Early exit condition}
				\State \Return ReconstructPath(previous,$\alpha$,$v$)
			\EndIf
			\State Q.enqueue($v$)
			\State S.add($v$)
			\State previous[$v$] $\gets u$
		\EndIf
	\EndFor
\EndWhile
\EndProcedure
\end{algorithmic}
\end{algorithm}

Note that, even though BFS will always find a path between two nodes if they are connected, the path produced is not optimal when we consider weights. The path found will be the shortest in terms of the number of steps, but when taking into account the weights of the edges, this algorithm will not give us the shortest distance path, as we can see in figure \ref{fig:bfs-fail}.

%TODO Put example images
\begin{figure}
\centering
\includegraphics[width=1\linewidth]{bfs-fail}
\caption{An example on how BFS fails to find an optimal path when we use a weighted graph. In this example, moving to a white tile has cost 1, while moving to a gray tile has cost 10, so clearly, going in a straight line is not the optimal path.}
\label{fig:bfs-fail}
\end{figure}

Also, the algorithm doesn't seem very efficient, since we can potentially explore lots of nodes on a very dense graph.

We will first address the optimality problem in the next section, and then we will focus on efficiency.


\section{Dijkstra}
Dijkstra's algorithm is also a well known graph theory algorithm, used to find optimal paths between nodes in a graph with positive weights. Now, instead of a simple queue, we use a priority queue\footnote{See \ref{annex:priorityqueue} for more information on priority queues.}. We keep exploring all unvisited neighbours of a node, but now we insert them in the priority queue using the distance of the edge, in a way that the node with the least distance from the source will be extracted first.

Normally, the algorithm only gets a starting node, and computes the optimal paths to all reachable nodes from the origin, but since we're only concerned about finding the path to one of the goal nodes, we will use the early exit condition to end the execution as soon as we expand a goal node. Later, we will see that this still gives us the optimal path.

\begin{algorithm}
\caption{Dijkstra's algorithm}
\label{alg:dijkstra}
\begin{algorithmic}[1]
\Procedure{Dijkstra}{$G$, $\alpha$, $T$}
\Require Graph $G = (V, E)$, directed or undirected; source node $\alpha \in V$; set of goal nodes $T \subset V$
\Ensure Given $u \in V$, previous[$u$] gives us the node come from to reach $u$ in the path computed, and d[$u$] gives us the distance to that node, if it has been explored
\State Q $\gets PriorityQueue()$
\State $S \gets Set()$
\State d $\gets Map()$ \Comment{Keeps track of the shortest distance to each node}
\State previous $\gets Map()$
\For {$u \in V$}
	\State d[$u$] $\gets \infty$
	\State previous[$u$] $\gets \varnothing$
\EndFor
\State Q.insert((0, $\alpha$))
\State d[$\alpha$] $\gets 0$
\While {not Q.empty()}
	\State u $\gets$ Q.removeMin()
	\State S.add(u)
	\If {$u \in T$} \Comment{Early exit condition}
		\State \Return ReconstructPath(previous,$\alpha$,$u$)
	\EndIf
	\For {$v \in$ $u$.neighbours()}
		\If {$v \in S$}
			\Continue \Comment{Already explored node}
		\EndIf
		\State alt $\gets$ d[$u$] $+ w((u, v))$
		\If {alt $<$ d[$v$]}
			\State d[$v$] $\gets$ alt
			\State Q.insert(($v$, alt))
			\State previous[$v$] $\gets u$
		\EndIf
	\EndFor
\EndWhile
\EndProcedure
\end{algorithmic}
\end{algorithm}

We will now prove the correctness of the algorithm without the early exit condition, that is, given a node $u \in V$, the algorithm always finds the optimal path between $u$ and all other nodes $v \in V$ such that $u$ and $v$ are connected. We will do so by induction on the visited set used in algorithm \ref{alg:dijkstra}, $S$. We will also write the distance computed by Dijkstra's algorithm between two nodes $u, v$ as $d_D(u, v)$.

\begin{lemma}
\label{lemma:dijkstra}
At any given step of the algorithm, $\forall s \in S,\ d_D(u, s) = \delta(u, s)$
\end{lemma}
\begin{proof}
If $|S| = 0$, the statement is trivially true.

If $|S| = 1$, it must be $S = \{u\}$, since $S$ only grows in size, but $d_D(u, u) = 0 = \delta(u, u)$.

Now let's assume we are in an arbitrary step, and let $s$ be the current node being explored, not yet added to $S$. Let $S' = S \cup \{s\}$. By inductive hypothesis, we know that $\forall t \in S,\ d_D(u, t) = \delta(u, t)$. Now we only need to show that $d_D(u, s) = \delta(u, s)$.

Suppose that there exists a path $Q$ from $u$ to $s$ such that
\[d(Q) < d_D(u, s)\]
We know that the path $Q$ starts in $S$ (since $u \in S$), but at some point has to leave $S$ (since $s \not\in S$). Let $e = (x, y) \in Q \subset E$ be the first edge that leaves $S$, that is, $x \in S$ but $y \not\in S$. Let $Q_x \subset Q$ be the edges of $Q$ up until and without including the edge $e$. Clearly,
\[d(Q_x) + d(x, y) \le d(Q)\]

By the induction hypothesis, $d_D(u, x) = \delta(u, x) \le d(Q_x)$. Therefore,
\[ d_D(u, x) + d(x, y) \le d(Q) \]

Clearly, $\delta(u, y) \le d_D(u, x) + d(x, y)$.

Since $y \not\in S$, and since Dijkstra uses a priority queue to select the next reachable node with minimum distance, we know that $d_D(u, s) \le d_D(u, y)$.

Combining the inequalities, we get that
\[ d_D(u, s) \le d_D(u, y) \le d_D(u, x) + d(x, y) \le d(Q) < d_D(u, s) \]
which is a contradiction.

Therefore, $d_D(u, s) = \delta(u, s)$.
\end{proof}

\begin{theorem}[Correctness of Dijkstra's algorithm]
\label{thm:dijkstra}
Let $G = (V, E)$ be a weighted directed graph. Let $u \in V$. Then, after running Dijkstra's algorithm with start node $u$, the following is true
\[ \forall v \in C_u,\  d_D(u, v) = \delta(u, v) \]
\end{theorem}
\begin{proof}
We know that, at the end of Dijkstra's algorithm, we'll have explored all the nodes connected to $u$. That is, $S = C_u$. For each $v \in C_u$, apply \ref{lemma:dijkstra} to get the wanted result.
\end{proof}

\begin{corollary}
Given a source node and a single goal node, Dijkstra's algorithm always finds an optimal path between the two nodes if it exists.
\end{corollary}
\begin{proof}
Let $G = (V, E)$, and let $\alpha, \beta \in V$ be the source and goal nodes, respectively. If $\beta \not \in C_\alpha$, there doesn't exist an optimal path. Suppose $\beta \in C_\alpha$. By Theorem \ref{thm:dijkstra}, after running Dijkstra, $d_D(\alpha, \beta) = \delta(\alpha, \beta)$, so an optimal path has been found.
\end{proof}

Like with BFS, we can modify the algorithm to use early exit to improve performance.
\begin{proposition}
Using Dijkstra's algorithm with early exit with start node $\alpha$ and a single goal node $\beta$ ensures that an optimal path from $\alpha$ to $\beta$ will be found.
\end{proposition}
\begin{proof}
With early exit, when we explore the goal node $\beta$ we will end the execution of the algorithm. Using \ref{lemma:dijkstra}, we know that $d_D(u, \beta) = \delta(u, \beta)$, so we already have the optimal path between the two nodes.
\end{proof}

%TODO Talk about what happens when there are negative weights. Name Bellman-Ford algorithm.
Note that Dijkstra's algorithm only works for graphs with positive weights. For graphs with negative weights, there are other less efficient algorithms which also find optimal paths, like the Bellman-Ford algorithm.

\section{Greedy Best-First search}
\label{section:greedy}
As we have seen in the last section, we now have an algorithm that can find the optimal path between any two nodes in a graph with positive weights. We will now start to worry about improving the performance of the algorithm.

\begin{definition}
Given any node $u \in V$, $t \in T$ is a \textbf{preferred goal node} of $u$ if the distance of an optimal path from $u$ to $t$ does not exceed the distance of any other path from $u$ to a node in $T$.
\end{definition}

Let's forget about optimality for a moment, and modify our pathfinding algorithm to use only an heuristic. An heuristic is a function
\begin{align*}
	h \colon V &\to \mathbb{R}\\
	u &\mapsto h(u)
\end{align*}
which gives an estimate of the distance from a node to one of its preferred goal nodes, which we compute without having to expand extra nodes, and varies with each type of problem we have.

We will talk more about heuristics later, but for now, let's consider what happens when we use heuristics instead of the actual distance of the paths between nodes, like we did in Dijkstra's algorithm, for the priority queue ordering. When using the heuristic as ordering, the node closest to the goal will be the first to be explored, not regarding the distance travelled so far.

\begin{algorithm}
\caption{Greedy Best-First search}
\label{alg:greedy}
\begin{algorithmic}[1]
\Procedure{GreedyBestFirstSearch}{$G$, $\alpha$, $T$}
\Require Graph $G = (V, E)$, directed or undirected; source node $\alpha \in V$; set of goal nodes $T \subset V$
\Ensure Given $u \in V$, previous[$u$] gives us the node come from to reach $u$ in the path computed, if it has been explored
\State Q $\gets PriorityQueue()$
\State $S \gets Set()$
\State previous $\gets Map()$
\For {$u \in V$}
	\State previous[$u$] $\gets \varnothing$
\EndFor
\State Q.insert((0, $\alpha$))
\While {not Q.empty()}
	\State $u \gets$ queue.removeMin()
	\State S.add($u$)
	\If {$u \in T$} \Comment{Early exit condition}
		\State \Return ReconstructPath(previous,$\alpha$,$u$)
	\EndIf
	\For {$v \in$ $u$.neighbours()}
		\If {$v \in S$}
			\Continue
		\EndIf
		\State Q.insert(($v$, $h(v)$))
		\State previous[$v$] $\gets u$
	\EndFor
\EndWhile
\EndProcedure
\end{algorithmic}
\end{algorithm}

%TODO Define tilemap as 2d discrete map
%TODO Define manhattan distance
%\begin{definition}
%The \textbf{Manhattan distance} between two points %TODO FINISH
%\end{definition}

%TODO Talk about the problems with the greedy search

\section{$A^*$}
\label{section:astar}
As we have seen, Dijkstra always gives us optimal paths, but it wastes a lot of time exploring a lot of nodes that are not in a promising direction. On the other hand, Greedy Best-First Search explores nodes in a promising direction, but does not produce optimal paths reliably.

The $A^*$ algorithm is a combination of both algorithms. It takes into account both the actual distance from the source to a node, and the estimated distance from the node to the goal. Unlike Dijkstra's algorithm, which works for positive weights including zero, $A^*$ only works with strictly positive weights.

\begin{definition}
If an algorithm $A$ always finds an optimal path between the source node and a preferred goal node, we say that $A$ is \textbf{admissible}.
\end{definition}

Given a source node $\alpha \in V$ and a set of goal nodes $T$, let $\hat{f}(u) = \hat{g}(u) + \hat{h}(u)$, where 
\begin{align*}
\hat{g}(u) = \delta(\alpha, u)\\
\hat{h}(u) = \min_{t \in T} \delta(u, t)
\end{align*}
 
Note that for any node $v$ in an optimal path between $\alpha$ and a preferred goal node $t$ of $\alpha$, $\hat{f}(v) = \hat{f}(\alpha) = \hat{f}(t) = \delta(\alpha, t)$. We will usually write this distance as $f^*$.

Let $g(u)$ and $h(u)$ be functions which give us estimates for $\hat{g}(u)$ and $\hat{h}(u)$ respectively.

\begin{definition}
Given a source node and a set of goal nodes in a graph $G = (V, E)$, we define the \textbf{score} as a function $f \colon V \to \mathbb{R}$ defined as
\begin{equation}
	f(u) = g(u) + h(u)
\end{equation}
where $g(u)$ is an estimate of the optimal distance from the source to the node $u$, and $h(u)$ is an estimate of the optimal distance from the node $u$ to one of its preferred goal nodes. We usually call $g$ the \textbf{g-score}, and $h$ the \textbf{h-score}.
\end{definition}

In $A^*$, a good choice for the g-score is using, for each node $u$, the cost of the path from the source to $u$ found so far by the algorithm, which is equivalent to the distance we kept updating in Dijkstra's algorithm. Note that this implies $g(u) \geq \hat{g}(u) \ \forall u \in V$. For the h-score, we use some heuristic, which will depend on the problem.

When a node is explored, its g-score, and thus its score will be updated. The pseudocode for the algorithm is shown in algorithm \ref{alg:astar}. As we can see, it's very similar to Dijkstra's, except that we now use the score for the priority queue ordering, instead of just the distance to that node.

\begin{algorithm}
\caption{$A^*$ algorithm}
\label{alg:astar}
\begin{algorithmic}[1]
\Procedure{AStar}{$G$, $\alpha$, $T$}
\Require Graph $G = (V, E)$, directed or undirected; source node $\alpha \in V$; set of goal nodes $T \subset V$
\Ensure Given $u \in V$, previous[$u$] gives us the node come from to reach $u$ in the path computed, and g[$u$] is the current g-score of the node
\State Q $\gets PriorityQueue()$
\State $S \gets Set()$
\State $g \gets Map()$
\State previous $\gets Map()$
\For {$u \in V$}
	\State g[$u$] $\gets \infty$
	\State previous[$u$] $\gets \varnothing$
\EndFor
\State Q.insert((0, $\alpha$))
\State $g[\alpha]$ $\gets 0$
\While {not Q.empty()}
	\State $u \gets$ Q.removeMin()
	\State $S$.add($u$)
	\If {$u \in T$} \Comment{Early exit condition}
		\State \Return ReconstructPath(previous,$\alpha$,$u$)
	\EndIf
	\For {$v \in u$.neighbours()}
		\If {$v \in S$}
			\Continue \Comment{Already explored node}
		\EndIf
		\State alt $\gets g[u] + w((u, v))$
		\If {alt $< g[v]$}
			\State $g[v] \gets$ alt \Comment{Update the g-score}
			\State $f = g[v] + h(v)$
			\State Q.insert(($v$, $f$))
			\State previous[$v$] $\gets u$
		\EndIf
	\EndFor
\EndWhile
\EndProcedure
\end{algorithmic}
\end{algorithm}

%TODO is this paragraph necessary...?
This new algorithm is faster than Dijkstra, but it only finds optimal reliably paths under a certain condition, which is what we will prove next.

\begin{definition}
A heuristic $h$ is said to be \textbf{admissible} if and only if, $\forall u \in V$, $h(u)$ never overestimates the real cost of moving from $u$ to a preferred goal node of $u$, that is, 
\[ \forall u \in V \ h(u) \le \hat{h}(u) \]
\end{definition}

We will prove that, with the score function we have constructed, an admissible heuristic implies that $A^*$ is admissible. Consider a graph $G = (V, E)$, a source node $\alpha$ and a set of goal nodes $T$, such that $\forall t \in T$, $t \in C_\alpha$. Consider the closed set $S$, which corresponds to the set of visited nodes in the algorithm, and the open set $O$, which are the nodes that haven't been explored yet.

\begin{lemma}
\label{lemma:astar1}
For any node $u \not \in S$ and for any optimal path $P$ from $\alpha$ to $u$, $\exists v \in O$ which is part of $P$ such that $g(v) = \hat{g}(v)$.
\end{lemma}
\begin{proof}
Consider an optimal path from $\alpha$ to $u$,
\[P = \{ (u_0 = \alpha, u_1), (u_1, u_2), \dots, (u_{n-1}, u_n = u) \}\]

If $v = \alpha$, which implies that $\alpha \in O$, so the algorithm hasn't completed the first iteration yet, then the lemma is trivially true since $g(\alpha) = \hat{g}(\alpha) = 0$.

Now suppose $\alpha \in S$. Let
\[ \Delta = \{ n \in S \  | \  n \text{ is part of } P,\  g(n) = \hat{g}(n) \} \]
$\Delta$ is clearly not empty since $\alpha \in \Delta$.

Let $n^* \in \Delta$ be the node that satisfies $d_P(\alpha, n^*) \leq d_P(\alpha, n) \  \forall n \in \Delta$. Since $w(e) > 0\  \forall e \in E$, $n^*$ is unique. Also, $n^* \neq u$ since $u \not \in S$. Let $v$ be the successor of $n^*$ in $P$. Note that it's possible that $v = u$.

Now, $g(v) \leq g(n^*) + w((n^*, v))$ by the definition of $g$, and since $n^* \in \Delta$, $g(n^*) = \hat{g}(n^*)$. Also, $\hat{g}(v) = \hat{g}(n^*) + w((n^*, v))$, because $P$ is an optimal path. Therefore, $g(v) \leq \hat{g}(v)$. But it is always true that $g(v) \geq \hat{g}(v)$. Therefore, $g(v) = \hat{g}(v)$, and by the definition of $\Delta$, $v \not \in \Delta$ implies $v \in O$.
\end{proof}

\begin{lemma}
\label{lemma:astar2}
Suppose the heuristic used by $A^*$ is admissible, and suppose $A^*$ has not finished its execution yet. Then, for any optimal path $P$ from $\alpha$ to a preferred goal node of $\alpha$, $\exists v \in O$ which is part of $P$ such that $f(v) \leq f^*$.
\end{lemma}
\begin{proof}
By lemma \ref{lemma:astar1}, $\exists v$ in $P$ with $g(v) = \hat{g}(v)$. Therefore, by the definition of the score function and the hypothesis,
\begin{align*}
f(v) &= g(v) + h(v)\\
&= \hat{g}(v) + h(v)\\
&\leq \hat{g}(v) + \hat{h}(v) = \hat{f}(v)
\end{align*}
Since $P$ is an optimal path, $\hat{f}(v) = f^* \  \forall v$ in the path $P$, which proves the lemma.
\end{proof}

\begin{proposition}
\label{prop:astar-termination}
Let $G = (V, E)$ be a graph with strictly positive weights, which can be infinite. If there exists a path from a source node $\alpha$ to a goal node in $T$, then $A^*$ terminates for every heuristic such that $h(u) \geq 0 \  \forall u \in V$.
\end{proposition}
\begin{proof}
Let
\[ \epsilon = \min_{e \in E} w(e) \]
By hypothesis, $\epsilon > 0$. For any node $u$ further than $M = f^*/\epsilon$ steps from $\alpha$, we have
\[ f(u) \geq g(u) \geq \hat{g}(u) > M \epsilon = f^* \]
where the last inequality comes from our definition of the g-score.

By Lemma \ref{lemma:astar2}, we see that there will always be a node $v \in O$ on an optimal path from the source to one of its preferred goal nodes such that $f(v) \leq f^* < f(u)$, so the algorithm will always pick it first, and therefore no nodes further than $M$ steps from $\alpha$ are ever explored.

Then, the only reason why the algorithm never terminates is because it's trapped in a loop where it repeatedly explores nodes that are less than $M$ steps away from $\alpha$. Let $V_M$ be the set of nodes accessible within $M$ or less steps from $\alpha$. Consider any node $u \in V_M$. There are only a finite number of paths from $\alpha$ to $u$ that only pass through nodes in $V_M$, and therefore $u$ can only be explored a finite number of times. We call this number $m_u$. Let
\[ m = \max_{u \in V_M} m_u \]
which is the maximum number of times any node can be explored. Then, after at most $|V_M| \cdot m$ expansions, all nodes in $V_M$ will be in the closed set. But we've seen that no nodes outside $V_M$ can be explored. Therefore, $A^*$ must terminate.
\end{proof}

\begin{theorem}[Admissibility of $A^*$]
\label{thm:astar}
If the heuristic is admissible, then $A^*$ is admissible.
\end{theorem}
\begin{proof}
There are only three possible cases in which $A^*$ doesn't find an optimal path: the algorithm terminates at a node which is not a goal node, terminates at a goal node but the path found isn't optimal, or fails to terminate at all. We've already seen in Proposition \ref{prop:astar-termination} that $A^*$ always terminates, even if the heuristic is not admissible. Let's consider the other two cases separately.

Consider the case where the algorithm terminates at a node which is not a goal node. Since we are using the early exit condition, which makes the algorithm terminate whenever it expands a goal node, the only possible way that the algorithm terminates without having found a goal node is if there wasn't any goal node in the open set to start with, or equivalently, no goal node is not connected to the start node, which contradicts our hypothesis.

Consider the case where the algorithm terminates at a goal node, but doesn't find an optimal path. Let $\beta$ be the goal node at which the algorithm terminates. Note that since the heuristic is admissible
\[ 0 \leq h(\beta) \leq \hat{h}(\beta) = 0 \  \Rightarrow \  h(\beta) = 0 \]
In this case, by the time the algorithm terminates, we have $f(\beta) = g(\beta) > f^*$ because the path found is not optimal. But by lemma \ref{lemma:astar2}, just before termination, there existed a node $u \in O$ on an optimal path with $f(u) \leq f^* < f(\beta)$, and since the algorithm uses a priority queue with the score, the node $u$ would have been selected for expansion instead of $\beta$, and the algorithm wouldn't have terminated.
\end{proof}

In the last section we proved that Dijkstra's algorithm always finds optimal paths when using a single goal node, but now we can easily extend this result when using a set of goal nodes.

\begin{corollary}[Admissibility of Dijkstra's algorithm]
Dijkstra's algorithm is admissible.
\end{corollary}
\begin{proof}
Dijkstra's algorithm is equivalent to $A^*$ with $h(u) = 0 \  \forall u \in V$. This heuristic is clearly admissible, so by Theorem \ref{thm:astar}, Dijkstra's algorithm is admissible.
\end{proof}

%TODO reference manhattan distance which will be defined in a future chapter
In figure \ref{fig:counterexample:admissible} we see a counterexample which proves that the converse of the Theorem is not true in general. Here, we are using the Manhattan distance as heuristic, which is an inadmissible heuristic if diagonal movement is allowed, since it will overestimate the true optimal distance between nodes. We can see that $A^*$ finds a path with distance $23$, which is not optimal because, in the same circumstances, Dijkstra finds a path with distance $16$.

\begin{figure}
\centering
\begin{subfigure}[b]{0.65\textwidth}
	\includegraphics[width=1\linewidth]{astar-fail}
	\caption{$A^*$}
\end{subfigure}
\begin{subfigure}[b]{0.65\textwidth}
	\includegraphics[width=1\linewidth]{astar-fail-dijkstra}
	\caption{Dijkstra}
\end{subfigure}
\caption{Pathfinding algorithms executed with diagonal movement allowed, and in the case of $A^*$, using the Manhattan distance as heuristic.}
\label{fig:counterexample:admissible}
\end{figure}

In figure \ref{fig:node-expansion} we can see a visual comparison of the number of nodes explored using algorithms \ref{alg:dijkstra}, \ref{alg:greedy} and \ref{alg:astar} in a single example. For algorithms \ref{alg:greedy} and \ref{alg:astar}, we have used the Manhattan distance as heuristic. We can see that Dijkstra explores many more nodes than the other two (it explores even more nodes that didn't fit in the image), while Greedy Best-First Search explores many fewer nodes but the path it finds isn't optimal, since the distance of the path it calculates is $35$, while Dijkstra and $A^*$ find paths of distance $33$.

\begin{figure}
\centering
\begin{subfigure}[b]{0.45\textwidth}
	\includegraphics[width=1\linewidth]{astar-nodes}
	\caption{$A^*$}
\end{subfigure}
\begin{subfigure}[b]{0.45\textwidth}
	\includegraphics[width=1\linewidth]{dijkstra-nodes}
	\caption{Dijkstra}
\end{subfigure}
\begin{subfigure}[b]{0.45\textwidth}
	\includegraphics[width=1\linewidth]{greedy-nodes}
	\caption{Greedy Best-First Search}
\end{subfigure}
\caption{Pathfinding algorithms executed with diagonal movement allowed, and in the case of $A^*$, using the Manhattan distance as heuristic.}
\label{fig:node-expansion}
\end{figure}


\chapter{Heuristics}
In the previous chapter we have proved that as long as the heuristic is admissible, $A^*$ produces optimal paths. But not any admissible heuristic will give us the same performance. We've seen that by taking the admissible heuristic $h(u) = 0 \  \forall u \in V$, the algorithm becomes equivalent to Dijkstra, so we haven't gained anything.

$A^*$ is part of a class of algorithms, which we will call $\mathcal{A^*}$, and by selecting the g-score and h-score functions we select one of the algorithms from the class. We normally only change the h-score, and use the g-score defined previously, but notice how if we make $g(u) = 0 \  \forall u \in V$, then we get the Greedy Best-First Search algorithm seen in section \ref{section:greedy}. However, unless we explicitly say otherwise, we will always use the g-score we defined in section \ref{section:astar} for our $A^*$ algorithms.

The class $\mathcal{A^*}$ is a subclass of a larger set of algorithms which are usually known as \textbf{informed} or \textbf{heuristic search algorithms}.

In this chapter, we will see that there are heuristics which also make $A^*$ optimal, meaning that we can reduce the number of nodes expanded with respect to other algorithms of $\mathcal{A^*}$.

\section{Consistent heuristics}

\begin{definition}
A heuristic $h$ is said to be \textbf{consistent} or monotone if and only if, $\forall u, v \in V$,
\begin{equation}
	h(u) \leq \delta(u, v) + h(v)\label{eq:consistency}
\end{equation}
and
\begin{equation}
	h(t) = 0 \  \forall t \in T
\end{equation}
where $T$ is the set of goal nodes.
The relation shown in \ref{eq:consistency} is called the triangle inequality.
\end{definition}

For example the heuristic defined as $h(u) = 0 \  \forall u \in V$ is trivially consistent.

\begin{proposition}
\label{prop:consistent}
Given an heuristic $h$,
\[ h \text{ is consistent} \Rightarrow h \text{ is admissible} \]
\end{proposition}
\begin{proof}
Let $T$ be the set of goal nodes. We will prove that $\forall u \in V$ such that it has a connected preferred goal node in $T$, $h(u) \leq \hat{h}(u)$, by induction on the number of steps from $u$ to its preferred goal node in an optimal path $P$.

Let $t$ be a preferred goal node of $u$. If there are 0 steps from $u$ to $t$ in $P$, then $u = t$, and $h(u) = h(t) = 0 \leq \hat{h}(u) = 0$.

Now suppose that $u$ is $k > 0$ steps away from $t$ in $P$. Let $v$ be the successor of $u$ in $P$, which is $k-1$ steps away from $t$. Therefore, since $h$ is consistent,
\[ h(u) \leq \delta(u, v) + h(v) \]
But by our inductive hypothesis, $h(v) \leq \hat{h}(v)$, and therefore
\[ h(u) \leq \delta(u, v) + \hat{h}(v) = \hat{h}(u) \]
since $P$ is an optimal path.
\end{proof}

The converse is not true in general, although in practice it can be difficult to find an admissible heuristic that is not also consistent. We will show this with a counterexample. Take the graph from figure \ref{fig:heuristic}, and let $C$ be the single goal node. Let's define the heuristic as
\begin{align*}
h(A) &= 4\\
h(B) &= 1\\
h(C) &= 0
\end{align*}
The heuristic is clearly admissible, since $h(A) \leq \hat{h}(A) = 4$, $h(B) \leq \hat{h}(B) = 3$, and $h(C) \leq \hat{h}(C) = 0$. But it is not consistent, since $h(A) > \delta(A, B) + h(B) = 1 + 1 = 2$.

\begin{figure}
\centering
\begin{tikzpicture}
	[scale=.8, auto, node_style/.style={circle, draw=blue, fill=blue!20},
	edge_style/.style={draw=black, ultra thick}]
	\node[node_style] (n1) at (-5,0) {A};
  	\node[node_style] (n2) at (0,0) {B};
  	\node[node_style] (n3) at (5,0) {C};
  	
  	\draw[edge_style] (n1) edge node{1} (n2);
  	\draw[edge_style] (n2) edge node{3} (n3);
\end{tikzpicture}
\caption{A weighted undirected graph.}
\label{fig:heuristic}
\end{figure}

The following result is of special interest, since it will allow us to design $A^*$ algorithms which don't require to re-open already closed nodes, simplifying the implementation, and will be useful in our proof of optimality.

\begin{proposition}
\label{prop:closed-optimal}
With a consistent heuristic, any node $u$ closed by $A^*$ satisfies
\[ g(u) = \hat{g}(u) \]
\end{proposition}
\begin{proof}
Remember that our definition of the g-score implied $g(v) \geq \hat{g}(v) \ \forall v \in V$.

Consider the state just before closing a node $u$, and suppose $g(u) > \hat{g}(u)$. Since $u$ is about to be closed, it means it's connected to the source node $\alpha$, and therefore an optimal path $P$ exists between $\alpha$ and $u$. Since $g(u) > \hat{g}(u)$, the algorithm hasn't found $P$ or any other optimal path. Since it did not find $P$, by Lemma \ref{lemma:astar1},
\[ \exists v \in O \mid g(v) = \hat{g}(v) \]
and such that $v$ is part of $P$.

If $u = v$, then the proof is finished. Suppose $u \neq v$. Then, we have
\[ \hat{g}(u) = \hat{g}(v) + \delta(v, u) = g(v) + \delta(v, u) \]
since $P$ is an optimal path from $\alpha$ to $u$, and $v$ is part of $P$. Then, by our supposition,
\[ g(u) > g(v) + \delta(v, u) \]
Adding the $h(u)$ to both sides,
\[ g(u) + h(u) > g(v) + \delta(v, u) + h(u) \geq g(v) + h(v) \]
where in the second inequality we have used equation \ref{eq:consistency}.

Note that this result is equivalent to
\[ f(u) > f(v) \]
which contradicts the fact that $u$ was selected for expansion, when $v$ was available and should have been picked first by the algorithm.
\end{proof}

By our definition of the g-score, Proposition \ref{prop:closed-optimal} implies that when a node is closed, the algorithm has found an optimal path from the source node to the closed node.

The following result is also very strong, as it tells us that nodes are closed in a monotonic order of their score value.

\begin{lemma}
\label{lemma:order}
Consider the ordered set of the nodes closed by $A^*$, 
\[ \{ \alpha = n1, n2, \dots, n_m \} \]
ordered by the time they were closed by the algorithm. If the heuristic is consistent, then, for $p, q \in \{ 1, \dots, m \}$,
\[ p \leq q \Rightarrow f(n_p) \leq f(n_q) \]
\end{lemma}
\begin{proof}
Let $u$ be a closed node, and $v$ the node closed just before $u$. Suppose the optimal path $P$ computed by $A^*$ from the source node $\alpha$ to $u$ does not go through $v$. This means that the algorithm selected $v$ for expansion when $u$ was also available, which means $f(v) \leq f(u)$, proving the Lemma.

Now suppose that the optimal path to $u$, $P$, does go through $v$. Then, $\hat{g}(u) = \hat{g}(v) + \delta(v, u)$. By Proposition \ref{prop:closed-optimal}, we know that $g(u) = \hat{g}(u)$ and $g(v) = \hat{g}(v)$. Then,
\begin{align*}
f(u) &= g(u) + h(u)\\
&= \hat{g}(u) + h(u)\\
&= \hat{g}(v) + \delta(v, u) + h(u)\\
&\geq \hat{g}(v) + h(v)\\
&= g(v) + h(v) = f(v)
\end{align*}
where we have used equation \ref{eq:consistency}.

Since this is valid for any two nodes that are adjacent in the sequence, then we can clearly see by induction that
\[ p \leq q \Rightarrow f(n_p) \leq f(n_q) \]
\end{proof}


\section{Optimality of $A^*$}
Choosing consistent heuristics has a practical benefit. Like we've seen in Proposition \ref{prop:closed-optimal}, with a consistent heuristic we can avoid having to re-open nodes which were already closed. We will prove that with a consistent heuristic, $A^*$ is also optimal. By proposition \ref{prop:consistent} we see that consistency is a stronger condition than admissibility, and therefore by theorem \ref{thm:astar}, with a consistent heuristic the algorithm will still find optimal paths.

\begin{definition}
Given two admissible algorithms $A$ and $B$, we say that $A$ \textbf{dominates} $B$ if and only if the set of nodes expanded by $A$ is a subset of the set of nodes expanded by $B$.
\end{definition}

\begin{definition}
Given two admissible algorithms $A$ and $B$, we say that $A$ \textbf{strictly dominates} $B$ if and only if $A$ dominates $B$ and $B$ does not dominate $A$.
\end{definition}

\begin{definition}
Given an heuristic search algorithm $A$, we say that $A$ is \textbf{no more informed} than $A^*$ if $A$ has access to the same heuristic information as $A^*$, but placing no restriction in how it uses it, and doesn't have any extra information that $A^*$ does not have about unvisited nodes.
\end{definition}

We can then treat the heuristic as a parameter of the graph, where each node $u$ has an assigned value $h(u)$, and all heuristic search algorithms no more informed than $A^*$ have access to this information.

We will also assume that all the algorithms discussed use a sequential approach to expanding nodes, only expanding nodes that have previously been seen from another node, and starting the expansion on the source node $\alpha$.

\begin{lemma}
\label{lemma:astar-opt}
Suppose we have an admissible heuristic, and suppose that $A^*$ has not terminated yet. Then, for every closed node $u$,
\[ f(u) \leq f^* \]
\end{lemma}
\begin{proof}
If $T$ is the set of goal nodes, then $u \not \in T$ since the algorithm has not terminated. By Lemma \ref{lemma:astar2}, at the time time the node $u$ was closed, there existed a node $v \in O$ which is part of an optimal path $P$ from the source node $\alpha$ to a preferred goal node of $\alpha$ such that $f(v) \leq f^*$.
But since $u$ was closed before $v$, we have
\[ f(u) \leq f(v) \leq f^* \]
\end{proof}

\begin{proposition}
\label{prop:surely}
If the heuristic is consistent, then $A^*$ expands every node reachable from the source node $\alpha$ with optimal distance strictly bounded by $f^*$.
\end{proposition}
\begin{proof}
Let $t$ be the goal node at which $A^*$ terminates its execution. Since $A^*$ is admissible by hypothesis, then $f(t) = f^*$. Since by Lemma \ref{lemma:order} $A^*$ closes nodes in non-decreasing order of their score, there can't exist a non-closed node $u$ with $f(u) < f(t) = f^*$. Therefore, all such nodes must have been closed by the time the algorithm closes $t$, and by Proposition \ref{prop:closed-optimal}, the distance found by the algorithm from the source node is optimal. Then, for any closed node $u$ with score strictly less than $f^*$, $f(u) = g(u) + h(u) = \hat{g}(u) + h(u) < f^*$ which implies $\hat{g}(u) < f^*$.
\end{proof}

\begin{definition}
We say that every node expanded with the property from Proposition \ref{prop:surely} is \textbf{surely expanded} by $A^*$. Given a source node $\alpha$, we will write the set of nodes surely expanded by $A^*$, that is, with its optimal distance from the source strictly bounded by $f^*$, as $N_{f^*}$.
\end{definition}

\begin{theorem}[Optimality of $A^*$]
\label{thm:astar-optimality}
Let $A$ be an admissible algorithm no more informed than $A^*$, and let $h$ be a consistent heuristic used by the algorithms. Then, $A$ will always expand all nodes surely expanded by $A^*$.
\end{theorem}
\begin{proof}
Let $G = (V, E)$ be a graph, and $\alpha \in V$ the source node with which we will execute the algorithms, and $T$ the set of goal nodes. Let $u$ be a node surely expanded by $A^*$, that is, $u \in N_{f^*}$. Suppose $A$ also halts when expanding a node with cost $f^*$.

Suppose $A$ does not expand $u$. Now, consider a new graph $G' = (V', E')$ which we create by taking $G$, and adding a new goal node $t$ and an edge $e = (u, t)$ with cost $w(e) = h(u) + \Delta$ where
\[ \Delta = \frac{1}{2} (f^* - \max \{ f(v) \mid v \in N_{f^*} \}) > 0 \]
Then, $V' = V \cup \{ t 	\}$ and $E' = E \cup \{ e \}$.

In this new graph, we see that when $A^*$ expands $u$, it will assign $t$ with a score of
\[ f(t) = g(t) + h(t) = g(t) = g(u) + w(e) = f(u) + \Delta \leq f^* - \Delta < f^* \]
Therefore, there will be a new solution path with cost at most $f^* - \Delta$, which will be found by $A^*$.

We have to see that in this new graph $G'$, consistency is maintained. Since we left the h-score unchanged for the nodes in $G$, consistency still holds for any pair of values in $G$ by hypothesis. Then, we only have to check that consistency holds for any pair of values containing the added node $t$. Since $t$ is a goal node, $h(t) = 0$, and all we have to check is that $\forall v \in V \  h(v) \leq \delta(v, t)$.

Suppose that $\exists v \in V$ for which $h(v) > \delta(v, t)$. Then,
\[ h(v) > \delta(v, t) = \delta(v, u) + w(e) = \delta(v, u) + h(u) + \Delta \]
which violates the heuristic's consistency in $G$, since $\Delta > 0$.

Since $A$ is not more informed than $A^*$, it will behave the same way in $G'$ as it did in $G$, therefore not expanding $u$ and failing to find the path to $t$ with cost lower than $f^*$, which contradicts the fact that $A$ is admissible.
\end{proof}

\begin{definition}
A \textbf{tie-breaking} rule is the rule that $A^*$ uses to choose the next node to expand when there is more than one node with the same score.
\end{definition}

We observe that in cases in which there are no non-goal nodes with score $f^*$, the set of nodes surely expanded by $A^*$ is actually the entire set of nodes expanded by $A^*$. This happens when $h(u) < \hat{h}(u)$ for every node $u$ in the graph, and in this case we say that the algorithm is not fully informed. If there are nodes with score equal to $f^*$ other than the preferred goal node, the algorithm may potentially have to expand all the nodes with such score until it finds the preferred goal node. Then, the optimality of the algorithm depends on the tie-breaking rule used.

The following result is useful when we can't find consistent heuristics for some specific problems. It tells us that by finding higher bounds of an admissible heuristic, we generally can improve efficiency when there is a single goal node. This efficiency is, of course, bounded by that of a consistent heuristic, as we proved in Theorem \ref{thm:astar-optimality}.

\begin{theorem}
\label{thm:dominating-heuristic}
Let $G = (V, E)$ be a graph, let $h, h'$ be two admissible heuristics, and let $A^*_h, A^*_{h'} \in \mathcal{A^*}$ be the $A^*$ algorithms using these heuristics. Suppose we only have a single goal node $t$. Then,
\[ h'(u) \geq h(u) \  \forall u \in V \Rightarrow A^*_{h'} \text{ dominates } A^*_h \]
\end{theorem}
\begin{proof}
Suppose $A^*_h$ does not expand a node $u \in V$. Then, it suffices to show that $A^*_{h'}$ does not expand $u$ either.

If $u$ is not expanded by $A^*_h$, then either $f(u) > f(t)$ or $f(u) = f(t)$ and the tie-breaking rule selected $t$ before $u$. Let's call $f'$ the score used by $A^*_{h'}$.

Since $h'(u) \geq h(u)$, $f'(u) = g(u) + h'(u) \geq g(u) + h(u) = f(u)$. Then, if $f(u) > f(t)$, we have $f'(u) > f'(t)$ and $u$ will not be expanded. If $f(u) = f(t)$ and $h(u) = h'(u)$, then $f'(u) = f'(t)$, and the same tie-breaking rule will still select $t$ before $u$, therefore not expanding $u$.
\end{proof}

Suppose there was more than one goal node in the premises of Theorem \ref{thm:dominating-heuristic}. Let $t$ be the preferred goal node found by $A^*_h$, and $t'$ the one found by $A^*_{h'}$. Since both heuristics are admissible, and both $t$ and $t'$ are preferred goal nodes, $f(t) = \hat{g}(t) = \hat{g}(t') = f'(t')$. In the case where $f(u) = f(t)$ and $h(u) = h'(u)$, we would have $f'(u) = f'(t')$, but the tie-breaking rule could expand $u$ before $t'$, so the result would not hold in this case.

Note that, in general, it is very difficult to find an heuristic such that $h(u) = \hat{h}(u) \  \forall u \in V$, so we usually have to settle for some good lower bound by approximating this value.

\section{Heuristics on grid maps}
Our simulator uses a grid map, so it is convenient to talk about some of the different heuristics we could use in this type of problem. In a grid graph, we have nodes uniquely identified by their coordinates, $(x, y)$, where $x,y \in \mathbb{Z}$.

\begin{definition}
The \textbf{Manhattan distance} between two nodes with coordinates $(x_0, y_0)$ and $(x_1, y_1)$ is defined as
\begin{equation}
|x_1 - x_0| + |y_1 - y_0|
\end{equation}
\end{definition}

If the grid only allows horizontal and vertical movement, then it's easy to see that the Manhattan distance between any two nodes $u, v$ is actually $\delta(u, v)$. Therefore, a good heuristic in this case is to use the Manhattan distance from any node $u$ to the goal node, since in this case $h(u) = \hat{h}(u)$.

\begin{definition}
The \textbf{Chebyshev distance} between two nodes with coordinates $(x_0, y_0)$ and $(x_1, y_1)$ is defined as
\begin{equation}
\max \{ |x_1 - x_0|, |y_1 - y_0| \}\label{eq:chebyshev}
\end{equation}
\end{definition}

If the grid allows diagonal movement, and moving diagonally has the same base cost as moving horizontally or vertically, then the Chebyshev distance between any two nodes $u, v$ is $\delta(u, v)$, and we can use it as the heuristic to get optimal efficiency.

Remember that, as we saw in figure \ref{fig:counterexample:admissible}, the heuristic that uses the Manhattan distance is not admissible in a grid that allows diagonal movement, and as a result the algorithm will not be admissible.

\begin{definition}
The \textbf{octile distance} between two nodes with coordinates $(x_0, y_0)$ and $(x_1, y_1)$ is defined as
\begin{equation}
\max \{ |x_1 - x_0|, |y_1 - y_0| \} + \sqrt{2} \min \{ |x_1 - x_0|, |y_1 - y_0| \}\label{eq:octile}
\end{equation}
\end{definition}

This distance is useful as an heuristic when the base cost of moving diagonally is $\sqrt{2}$, using the Pythagorean theorem on the base costs of moving horizontally and vertically which are $1$.

The Chebyshev and octile distances are actually special cases of a more general distance called the diagonal distance.

\begin{definition}
The \textbf{diagonal distance} between two nodes with coordinates $(x_0, y_0)$ and $(x_1, y_1)$ is defined as
\begin{equation}
D \max \{ |x_1 - x_0|, |y_1 - y_0| \} + (D' - D) \min \{ |x_1 - x_0|, |y_1 - y_0| \}\label{eq:diagonal-distance}
\end{equation}
where $D$ is the cost of moving horizontally and vertically, and $D'$ is the cost of moving diagonally.
\end{definition}

Observe that equation \ref{eq:chebyshev} is obtained by setting $D = D' = 1$ in equation \ref{eq:diagonal-distance}, and equation \ref{eq:octile} is obtained by setting $D = 1$ and $D' = \sqrt{2}$.

In all these cases, we could use the Euclidean distance between the points as heuristics, which is admissible, but the alternatives presented always dominate over it by Theorem \ref{thm:dominating-heuristic}.


\chapter{$A^*$ variants}

In the last chapters we have mainly focused on the $A^*$ algorithm. But the addition rule used by the score, $f = g + h$, is not the only way to use the heuristic information we have, and we could also tweak the way nodes are expanded, giving rise to new algorithms, fundamentally different from $A^*$, but still close enough to be considered part of the same family. In this chapter we will present some variants of the $A^*$ algorithm, explaining how they fundamentally change from the $A^*$ we have seen, and the benefits that using them may give us.


\section{$IDA^*$}
Iterative Deepening $A^*$, or $\bm{IDA^*}$, is a variant of $A^*$ based on iterative deepening search, which is a modification of Depth-First Search that finds the best depth limit. %TODO talk about DFS in the annex


%TODO make chapter on tie-breaking rules
%TODO make chapter on applications: 8-puzzle? route finding, AI, etc...

\begin{appendices}

\chapter{Data structures}

In this appendix we will briefly explain the data structures we used in our algorithms. We will only explain them on the interface level, and give the complexity of each of its operations. There are many different ways to implement these structures, but the details on the implementation and justification of the complexity are out of the scope of this article, but the information can be found in \emph{Introduction to Algorithms} by Thomas H. Cormen \textit{et al.}, or any other data structures or algorithmics book.

\section{Queue}
\label{annex:queue}
In algorithms \ref{alg:bfs} and \ref{alg:bfs_early_exit} we use a queue. A queue has two methods that modify the data in the structure:
\begin{itemize}
\item \emph{enqueue}: Adds an item at the back of the queue.
\item \emph{dequeue}: Removes the item in front of the queue. If there are no items in the queue, it does nothing.
\item \emph{empty}: Tells us whether the queue is empty or not.
\end{itemize}
For this reason, we say that a queue is a FIFO (first in, first out) structure, meaning that we will remove items from the queue in the same order as we inserted them. We always insert items from the back, and remove them from the front.

We also have a method, \emph{front}, which returns the item in the first position in the queue, but doesn't remove it.

All these methods have a time complexity of $O(1)$.

\section{Priority Queue}
\label{annex:priorityqueue}
In algorithms \ref{alg:dijkstra}, \ref{alg:greedy} and \ref{alg:astar} we use a priority queue. The priority queue works similarly to a queue, but each item also has a priority. This priority determines the order in which the items will be removed from the queue. In all our algorithms, we used a minimum priority queue, meaning that we associated a priority value to each item, and the item with minimum value would be served first. This is crucial for the correct functionality of all of our algorithms that use a priority queue. In Dijkstra, we used the current best distance to each node as the priority. In Greedy Best-First Search, we used the heuristic. And in $A^*$ we used the score function.

We use the following methods from the priority queue:
\begin{itemize}
\item \emph{insert}: Takes a pair of item and priority, and inserts the item into the right place in the priority queue. This method has a time complexity of $O(\log n)$ where $n$ is the number of elements in the priority queue.
\item \emph{removeMin}: Returns the item in the front of the priority queue, that is, the item with the smallest priority value (since it's a minimum priority queue), and removes it from the priority queue. This method has a time complexity of $O(1)$.
\item \emph{empty}: Tells us whether the queue is empty or not. Its complexity is $O(1)$.
\end{itemize}

We use the priority queue to get the node with minimum score (or distance in the case of Dijkstra, and h-score in the case of Greedy Best-First search) as the next node to be explored and closed. We also use this fact in many of the proofs.

Note that there are other implementations of the priority queue in which insertion is $O(1)$ and removing the item with minimum value is $O(\log n)$, but in the end it doesn't matter which one we use.

\section{Set}
A set is an unordered list of unique items, but differs in implementation from a regular array, since it's usually implemented using a hash map, so lookup is much faster. We use mainly two operations on the set: the \emph{add} method, which inserts an item into the set, and the \emph{lookup} method (which we represented using mathematical notation in our conditions), which determines whether an item is part of the set or not. Both of these operations have a time complexity of $O(1)$.

\section{Map}
A map is a structure that stores data in pairs of key and value. Once we have some data in it, we look up using a key, which retrieves its corresponding value. In all of our algorithms, we used maps to store the node we came from for each visited node. In Dijkstra and $A^*$, we also used maps to store the current best distance from the source to each node, or equivalently the g-score. A map has two main methods: one to insert data and one to look up data. We have represented this using square bracket notation:
\begin{itemize}
\item To insert or update data in a map $d$, we use $d[key] = value$. If the key did not exist in the map, it will be added with the given value. If the key already existed in the map, its value will be updated to the given value.
\item To look up data in a map $d$, we use $d[key]$, which would return the value associated to the given key. If the key was not in the map, this operation is illegal and the program should throw an error. In our case, we always initialize the maps at the beginning of the algorithms, so we can never run into this kind of problem.
\end{itemize}
Inserting and looking up data in a map has a time complexity of $O(1)$.

\end{appendices}

\end{document}
